%!TEX root = main.tex
We slightly adapt the notation used in \cite{khammash1990stability}.  We use Latin letters to denote vectors and matrices, e.g., $Ax = b$, and bold-face Latin letters to denote signals and operators, e.g., $\tf x = (x_t)_{t=0}^\infty$, and $\tf y = \tf G \tf u$.  We let $\ell_\infty$ denote the space of all bounded sequences of real numbers, i.e., $\tf x = (x_t)_{t=0}^\infty$ if and only if $\sup_t |x_t|<\infty$, in which case we define $\linfnorm{\tf x} = \sup_t |x_t|$.  Similarly, we let $\ell^q_\infty$ denote the space of all $q$-tuples of elements of $\ell_\infty$: if $\tf x = (\tf x^1, \dots, \tf x^q) \in \ell^q_\infty$, then $\linfnorm{\tf x} = \max_{i=1,\dots, q}\linfnorm{\tf x^i}$.  We also define the extended space $\ell^{q,e}_\infty$ which is equal to the space of all $q$-tuples of sequences of real numbers.  We let $\tf S_+$ denote the right shift operator such tht if $\tf x = (x_t)_{t=0}^\infty$, then $\tf S_+\tf x = (0,x_0, x_1, \dots)$.  Similarly, we let $\tf S_-$ denote the left shift operator such that $\tf S_- \tf x = (x_1,x_2, \dots)$.  Hence $S_-S_+ = I$, but in general $S_+ S_- \neq I$.  We let $\LTV^{p,q}$ be the space of all bounded linear causal operators mapping $\ell^q_\infty \to \ell^p_\infty$, and broadly refer to all such operators as $\ell_\infty$-stable.  If $\tf R \in \LTV^{p,q}$, then $\linffnorm{\tf R} := \sup_{\linfnorm{x}\leq 1}\linfnorm{\tf R \tf x}$, which is the induced operator norm.  Note that each $\tf R \in \LTV^{p,q}$ can be completely characterized by its block lower-triangular pulse response matrix.  We denote by $\LTI^{p,q}$ the subspace of $\LTV^{p,q}$ consisting of time-invariant operators.