%!TEX root = main.tex
Robust control seeks to explicitly account for the unavoidable mismatch between a mathematical model of a system, and the behavior of the system itself.  In the context of linear control systems, robust control techniques \cite{khammash1990stability,dahleh1994control,zhou1996robust} have proven invaluable in applications ranging from process control to aerospace engineering.  A theme in robust control is the tradeoff between how easily necessary and sufficient conditions for robust stability and performance can be obtained, and how detailed (conversely conservative) the description of model uncertainty is.  Indeed, many of the celebrated tools from robust control, such as the structured singular value (see \cite{packard1993complex}), structured small gain theorems (see Ch 7.2 of \cite{dahleh1994control}), and integral quadratic constraints (see \cite{megretski1997system}), seek to optimally navigate this tradeoff.  However, as of yet, few of these results have been extended in a convincing manner to the large-scale distributed setting.

Although there is a rich and increasingly mature body of work tackling distributed optimal control problems exists (see \cite{2006_Rotkowitz_QI_TAC, 2012_Mahajan_Info_survey, wang2019system,zheng2019equivalence}, and references therein), some of which have robust control interpretations with respect to unstructured bounded perturbations (e.g., \cite{matni2014distributed,lessard2014state,rosinger2017structured}), to the best of our knowledge no tight conditions for robust performance of large-scale distributed controllers that extend beyond these simple extensions exist.  In this paper, we extend the System Level Synthesis (SLS) parameterization \cite{anderson2019system,wang2019system} of internally stabilizing controllers to a class of systems described by causal bounded linear operators, and show how this parameterization can be used to make connections to classical robust synthesis techniques \cite{khammash1990stability,dahleh1994control}.  We then exploit this connection to show that these necessary and sufficient conditions are equally applicable when additional delay, sparsity, and locality constraints are imposed on the system responses and controller implementation.  To the best of our knowledge, these are the first such necessary and sufficient conditions that are applicable to large-scale uncertain systems and distributed controllers.  In particular, our contributions are:
\begin{itemize}
\item A generalization of the SLS parameterization of stabilizing state-feedback controllers for finite-dimensional LTI systems \cite{wang2019system} to systems with dynamics described by bounded and causal linear operators, showing that affine constraints on certain system response design variables are necessary and sufficient for them to be achievable by a causal linear controller;
\item A generalization of the robustness result of \cite{matni2017scalable} showing that the generalized SLS parameterization is stable with respect to perturbations away from the aforementioned achievability subspace, as well as an explicit characterization of the effects of these perturbations on the actual closed loop behavior achieved by the correspondingly perturbed controller implementation;
\item The formulation and solution of robust performance problem in terms of system response variables for a linear-time-invariant dynamical system subject to bounded and causal linear uncertainty that naturally allows for delay, sparsity, and locality structure to be imposed on the system response and corresponding controller implementation;
\end{itemize}

The rest of the paper is structured as follows: in Section 2, we introduce notation and basic definitions of stability and well-posedness.  In Section 3, we review the SLS parameterization for finite-dimensional LTI systems, and generalize it to a richer class of systems with dynamics described by bounded linear operators.  In Section 4, we consider a robust version of the generalized SLS parameterization derived in Section 3, and provide necessary and sufficient conditions for robust performance terms of \emph{convex} constraints on the system responses.  In Section 4.1, we highlight how these convex constraints can naturally be integrated with structural constraints on the system responses, such as delay, sparsity, and locality constraints, while still preserving the necessity and sufficiency of the identified conditions.  We end with conclusions and discussions of future work in Section 5.