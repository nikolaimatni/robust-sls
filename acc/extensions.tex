%!TEX root = main-ugly.tex
%In this section, we discuss to extensions to the robust performance results presented in the previous section.

\subsection{Robust Performance Guarantees for Large-Scale Distributed Control}
\label{sec:extensions}
In previous work \cite{wang2014localized,wang2016localized,wang2018separable}, it was shown that for LTI dynamical systems \eqref{eq:lti-dynamics} defined by structured (i.e., sparse) matrices $(A,B)$, imposing \emph{locality constraints} on the system responses $\{\Phix,\Phiu\}$, i.e., imposing that $\{\Phix,\Phiu\} \in \mathcal{S}$, for $\mathcal{S}$ a suitably defined structure inducing subspace constraint, leads to distributed controllers that enjoyed scalable synthesis and implementation complexity.  Although formally defining these concepts is beyond the scope of this paper, we note that such conditions can be easily enforced on the solution of the robust performance conditions \eqref{eq:constraints} by additionally imposing that $\{\Phixh,\Phiuh\}\in\mathcal{S}$.   Under suitable assumptions on the structure of the cost matrices $(C,D)$ (e.g., that $(C,D)$ are block-diagonal), the resulting problem satisfies a notion of \emph{partial separability}, c.f. \S IV of \cite{wang2018separable}, which allows for the problem to be solved at scale using tools from distributed optimization.  

We emphasize that the conditions identified in \eqref{eq:constraints} remain necessary and sufficient when additional structure is imposed on the system responses $\{\Phixh,\Phiuh\}$ so long as the dynamic perturbations $(\DA,\DB)$ remain unstructured.  In particular, Theorem 6.3 of \cite{khammash1990stability} is applicable to any augmented plant $\tf M \in \RHinf$ -- this condition holds true for the augmented plant defined in equation \eqref{eq:bigM} even when any additional constraints are imposed on $\Phih$.  To the best of our knowledge, these are the first such necessary and sufficient conditions for robust performance that are applicable to large-scale distributed systems.  An exciting direction for future work will be to explore the consequences of locality in the system responses $\{\Phixh,\Phiuh\}$ on necessary and sufficient conditions for robust performance when the perturbations $(\DA,\DB)$ are further constrained to respect the topology of the nominal system, as defined by the support of $(\hat A, \hat B)$.
%
%\subsection{Sub-Optimality Guarantees for Robust Control}
%In \cite{dean2017sample}, it was shown\footnote{The result in \cite{dean2018sample} is for a robust $\mathcal{H}_2$ problem, but the derivation goes through nearly as is for $\mathcal{L}_1$ problem considered here.} that an upper bound to the robust performance problem considered here can be computed by solving
%\begin{equation}
%\begin{array}{rl}
%\min_{\tau \in [0,1)}&\frac{1}{1-\tau}\min_{\Phix,\Phiu}\linffnorm{\begin{bmatrix} C & D \end{bmatrix} \begin{bmatrix} \Phixh \\ \Phiuh \end{bmatrix}} \\
%\text{subject to} & \begin{bmatrix} zI-\hat A & - \hat B \end{bmatrix}\begin{bmatrix}\Phixh \\ \Phiuh \end{bmatrix} = I, \ \linffnorm{\begin{bmatrix} \Phixh \\ \Phiuh \end{bmatrix}} \leq \frac{\tau}{\epsilon}, \\
%&\Phixh, \Phiuh \in \frac{1}{z}\RHinf,
%\end{array}
%\end{equation}
%and further, it was shown that $\epsilon$ is such that $\epsilon(1+\linffnorm{\K_\star})\linffnorm{(zI-(A+B\K_\star))^{-1}}\leq 1/5$, for $\K_\star$ the optimal co