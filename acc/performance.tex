%!TEX root = main-ugly.tex
We now use the tools developed in the previous section to identify necessary and sufficient conditions for the robust stability and robust performance of a system \eqref{eq:dynamics} subject to bounded perturbations in its $\A$ and $\B$ operators.  In particular consider the system
\begin{equation}
\tf x = \Sp(\A_0 + \DA)\x + \Sp(\B_0 + \DB)\uu + \Sp \w,
\label{eq:uncertain-dynamics}
\end{equation}
where $\A_0 = \mathrm{blkdiag}(\hat A,\hat A,\dots)$ and $\B_0 = \mathrm{blkdiag}(\hat B, \hat B,\dots)$ are memoryless LTI operators defining a nominal LTI system $x_{t+1} = \hat Ax_t + \hat Bu_t + w_t$, and $\DA$ and $\DB$ are $\ell_\infty$-stable and satisfy 
\begin{equation}
\linffnorm{[\DA,\, \DB]}\leq \eps.
\label{eq:eps-bound}
\end{equation}
%We first identify SLS based necessary and sufficient conditions for robust stability, and then build upon those to formulate a robust performance problem.  

We consider the following robust control problem: find a LTI controller $\tf K : \ell^n_{\infty,e} \to \ell^p_{\infty,e}$, using only the nominal dynamics $(\hat A,\hat B)$ and uncertainty bound $\eps$, such that the dynamics \eqref{eq:uncertain-dynamics} in closed loop with the control law $\uu = \K \x$ is $\ell_\infty$-stable for all admissible uncertainty realizations $(\DA,\DB)$ satisfying \eqref{eq:eps-bound}.  To do so, we recognize that for any LTI $\{\Phixh,\Phiuh\}$ satisfying the LTI achievability constraints \eqref{eq:lti-achievable} defined by $(\hat A, \hat B)$, we have that
\begin{multline}
\begin{bmatrix} I - \Sp\A_0 - \Sp\DA & - \Sp\B_0 - \Sp\DB \end{bmatrix}\begin{bmatrix} \Phixh \\ \Phiuh \end{bmatrix} \\= \begin{bmatrix} I - \Sp\A_0 & - \Sp\B_0 \end{bmatrix}\begin{bmatrix} \Phixh \\ \Phiuh \end{bmatrix}\\ - \begin{bmatrix} \Sp\DA & \Sp\DB \end{bmatrix}\begin{bmatrix} \Phixh \\ \Phiuh \end{bmatrix}\\ = \Sp\left(I - \Sp\begin{bmatrix} \DA & \DB \end{bmatrix}\begin{bmatrix} \Phixh \\ \Phiuh \end{bmatrix} \right)
\end{multline}
where the final inequality follows from the assumption that $\{\Phixh,\Phiuh\}$ satisfy the LTI achievability constraints \eqref{eq:lti-achievable}, or equivalently \eqref{eq:osls-achievable}, defined in terms of the dynamics $(\hat A, \hat B)$.  Noting that
\begin{equation}
\Dh := \Sp\begin{bmatrix} \DA & \DB \end{bmatrix}\begin{bmatrix} \Phixh \\ \Phiuh \end{bmatrix},
\end{equation}
is a strictly causal $\ell_\infty$-stable operator, we conclude by Theorem \ref{thm:robust-sls} that the controller implementation \eqref{eq:realization} defined in terms of the LTI operators $\{\Phixh,\Phiuh\}$ achieves the following closed loop behavior when applied to the uncertain dynamics \eqref{eq:uncertain-dynamics}:
\begin{equation}
\begin{bmatrix}\x \\ \uu \end{bmatrix} = \begin{bmatrix} \Phixh \\ \Phiuh \end{bmatrix}(I-\Dh)^{-1}\tf w.
\label{eq:dhat-response}
\end{equation}
Further this control law is internally stabilizing if and only if $(I-\Dh)^{-1}$ is $\ell_\infty$-stable.  

Define the controlled output signal to be 
\begin{equation}
\tf z = \tf C\x + \tf D\uu,
\label{eq:z-output}
\end{equation}
for $\tf C = \mathrm{blkdiag}(C,C,\dots)$ \& $\tf D = \mathrm{blkdiag}(D,D,\dots)$ user specified cost matrices,\footnote{For simplicity, we assume $\tf C$ and $\tf D$ to be memoryless and LTI, however our results are equally applicable when the controlled output is defined in terms of LTI  filters $\tf C(z)$ and $\tf D(z)$.} and consider the goal of minimizing the $\ell_\infty\to\ell_\infty$ induced gain from $\tf w \to \tf z$ of the uncertain system \eqref{eq:uncertain-dynamics}.  To lighten notation going forward, we let
\begin{multline}
 \Phih := \begin{bmatrix}\Phixh \\ \Phiuh \end{bmatrix}, \, \D := \frac{1}{\eps}\Sp\begin{bmatrix} \DA & \DB \end{bmatrix},\\ \tf Q := \begin{bmatrix} \tf C & \tf D \end{bmatrix}, \, \ Z_{AB} := \begin{bmatrix} zI-\hat A & - \hat B\end{bmatrix}.
\label{eq:notation}
\end{multline}

We can then pose the robust performance problem for a specified performance level $\gamma \geq 0$ as finding LTI operators $\{\Phixh,\Phiuh\}$ that satisfy 
\begin{equation}\label{eq:robust-perf1}
\begin{aligned}
&{ \bignorm{\Q\Phih \left(I-\eps\D\Phih\right)^{-1} }_{\ell_\infty \to \ell_\infty}} \leq \gamma \\
&Z_{AB}\Phih = I,\ \Phih \in \frac{1}{z}\RHinf, \\
& \left(I-\eps\D\Phih\right)^{-1} \text{ is $\ell_\infty$-stable}
\end{aligned}
\end{equation}
for all strictly causal $\D$ satisfying $\linffnorm{\D}\leq 1$, where we have combined equations  \eqref{eq:dhat-response}, \eqref{eq:z-output}, and \eqref{eq:notation} to derive the robust performance bound condition.  It then follows that the robust performance problem \eqref{eq:robust-perf1} is equivalent to finding an LTI operator $\Phih$ satisfying
\begin{equation} \label{eq:robust-perf}
\begin{aligned}
& \linffnorm{\frac{1}{\gamma}\Q\Phih + \frac{1}{\gamma}\Q\Phih\D(I-(\eps\Phih)\D)^{-1}(\eps\Phih)} \leq 1 \\
& Z_{AB}\Phih = I,\ \Phih \in \frac{1}{z}\RHinf,\   (I-\Phih\D)^{-1} \text{ is $\ell_\infty$-stable}\\
\end{aligned}
\end{equation}
for all strictly causal $\D$ satisfying $\linffnorm{\D}\leq 1$, where we have used that $(I-\Delta\Phih)^{-1} = I + \Delta(I-\Phih\Delta)^{-1}\Phih$ and that $(I-\tf{GH})^{-1}$ is $\ell_\infty$-stable if and only if $(I-\tf{HG})^{-1}$ is $\ell_\infty$-stable (see Proposition 1, \cite{khammash1990stability}) to recast the expression \eqref{eq:robust-perf1} in a form that matches the linear-fractional-transform (LFT) structure studied in \cite{khammash1990stability,dahleh1994control}.

We can therefore leverage the equivalence between robust stability and performance (see Theorem 5.1, \cite{khammash1990stability}) to conclude that $\Phih$ satisfies the robust performance conditions \eqref{eq:robust-perf} for all $\linffnorm{\D}\leq 1$ if and only if the augmented LTI system\footnote{Note that we can relax the requirement that $\D$ be strictly causal because the nominal plant is LTI.}
\begin{equation}
\tf M = \begin{bmatrix} \frac{1}{\gamma}\Q\Phih & \frac{1}{\gamma}\Q\Phih \\ \eps\Phih & \eps\Phih \end{bmatrix}
\label{eq:bigM}
\end{equation}
is robustly stable for all structured perturbations $\tilde{\D}$ 
\begin{multline}
\tilde{\D} = \mathrm{blkdiag}\left(\D_1,\D_2\right), \, \D_1, \D_2 \in \LTV, \\ \linffnorm{\D_1} \leq 1, \linffnorm{\D_2} \leq 1.
\end{multline}

The necessary and sufficient conditions for robust stability of the resulting two-block problem can be derived as a special case of Theorem 6.3 of \cite{khammash1990stability}.  The particular case of an  augmented LTI system $\tf M$ satisfying $\tf M_{11} = \tf M_{12}$ and $\tf M_{21} = \tf M_{22}$, as is the case for our problem \eqref{eq:bigM}, is addressed in Ch 8.3 of \cite{khammash1990stability}, where a similarly structured augmented system \eqref{eq:bigM} arises in the context of bounding output sensitivity in the presence of output perturbations.  The necessary and sufficient conditions specified in Theorem 6.3 of \cite{khammash1990stability} reduce to the following \emph{convex} constraints on the system response $\Phih$
\begin{equation} \label{eq:constraints}
\begin{aligned}
&Z_{AB}\Phih = I, \, \Phih \in \frac{1}{z}\RHinf, \\
& \linffnorm{\Q\Phih} + \gamma\linffnorm{\eps\Phih} < \gamma.
\end{aligned}
\end{equation}

\begin{remark}
The above results apply equally to the $\ell_1 \to \ell_1$ induced norm, as it is easily checked to equal the $\ell_\infty \to \ell_\infty$ induced norm of the transpose system (see \cite{matni2017scalable}).
\end{remark}
\begin{remark} Although the constraints \eqref{eq:constraints} are in general infinite-dimensional due to the transfer matrix $\Phih$, principled finite-dimensional approximations, some of which enjoy provable sub-optimality guarantees, are available \cite{matni2017scalable,anderson2019system,dean2017sample}.  Further, for the $\mathcal{L}_1$ problem considered here, the resulting optimization problem is quasi-convex, and can be posed as a linear program (LP) for a fixed $\gamma$, thus enjoying favorable computational complexity properties.
\end{remark}
%It then follows that by bisecting on $\gamma$, e.g., by using golden search, we can find a performance level $\gamma$, and corresponding system responses and controller, satisfying $\gamma \leq \gamma_\star + \epsilon$ in $O\log_2(1/\epsilon)$ iterations, for $\gamma_\star$ the smallest $\gamma$ such that the set defined by \eqref{eq:constraints} is non-empty.
%Thus, an LTI controller $\K=\Phiuh\Phixh^{-1}$ is robustly stabilizing for system \eqref{eq:uncertain-dynamics} if and only if the operators $\left\{\Phixh,\Phiuh\right\}$ satisfy

