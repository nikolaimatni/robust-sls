\documentclass[11pt]{article}

\usepackage{graphicx, amsmath,amsthm,amssymb,url}

\usepackage[labelfont=bf]{subcaption}
\usepackage[labelfont=bf]{caption}
\usepackage{algorithm,algcompatible,mathtools}
\usepackage[multidot]{grffile}
\usepackage[normalem]{ulem} % just for strikeout

\usepackage{fullpage,etoolbox}
\usepackage{color}
\usepackage[numbers,sort,compress]{natbib}

 
\newtheorem{theorem}{Theorem}
\numberwithin{theorem}{section}
\newtheorem{lemma}[theorem]{Lemma}
\newtheorem{proposition}[theorem]{Proposition}
\newtheorem{coro}[theorem]{Corollary} 
\newtheorem{remark}{Remark}

\numberwithin{equation}{section}

\newtoggle{draft}
%\toggletrue{draft}
\togglefalse{draft}

\newcommand{\todo}[1]{
\iftoggle{draft}{
\vspace{5 mm}\par \noindent \marginpar{\textsc{ToDo}}
\framebox{\begin{minipage}[c]{0.95 \columnwidth} \tt #1
\end{minipage}}\vspace{5 mm}\par}{}}

\newcommand{\propchange}[1]{
\iftoggle{draft}{
	{\leavevmode\color{blue}{#1}}}{\leavevmode{#1}}}

\newcommand{\propdelete}[2]{
\iftoggle{draft}{
\textcolor{blue}{\sout{#1} {#2}}}{#2}}


\newcommand{\minimize}{\mbox{minimize}}
\newcommand{\maximize}{\mbox{maximize}}
\newcommand{\st}{\mbox{subject to}}
\newcommand{\statespace}[4]{\left[ \begin{array}{c|c} #1 & #2 \\ \hline #3 & #4 \end{array} \right]}
\newcommand{\tf}[1]{\boldsymbol{#1}}
\newcommand{\Ah}{\widehat{A}}
\newcommand{\Ahat}{\Ah}
\newcommand{\Bhat}{\Bh}
\newcommand{\Bh}{\widehat{B}}
\newcommand{\Kh}{\widehat{K}}
\newcommand{\Jh}{\widehat{J}}
\newcommand{\Phixh}{\hat{\tf \Phi}_x}
\newcommand{\Phiuh}{\hat{\tf \Phi}_u}
\newcommand{\Phix}{\tf\Phi_x}
\newcommand{\Phiu}{\tf\Phi_u}
\newcommand{\Phih}{\hat{\tf \Phi}}
\newcommand{\Dh}{\hat{\tf{\Delta}}}
\newcommand{\D}{{\tf{\Delta}}}
\newcommand{\DA}{\D_{\A}}
\newcommand{\DB}{\D_{\B}}
\newcommand{\trueA}{A}
\newcommand{\trueB}{B}
\newcommand{\trueK}{K_\star}
\newcommand{\A}{\tf A}
\newcommand{\B}{\tf B}
\newcommand{\K}{\tf K}
\newcommand{\Q}{\tf Q}
\newcommand{\x}{\tf x}
\newcommand{\xhat}{\tf{\hat x}}
\newcommand{\uu}{\tf u}
\newcommand{\w}{\tf w}
\newcommand{\what}{\hat{\tf w}}
\newcommand{\eps}{\varepsilon}

\DeclareMathOperator*{\argmin}{arg\!\min}
\DeclareMathOperator*{\argmax}{arg\!\max}
\DeclareMathOperator*{\sgn}{sgn}
\DeclareMathOperator*{\supp}{supp}
\DeclareMathOperator*{\rank}{rank}
\DeclareMathOperator*{\diag}{diag}
\DeclareMathOperator*{\Tr}{\mathbf{Tr}}
\DeclareMathOperator*{\image}{Im}
\DeclareMathOperator*{\nullspace}{Kern}
\DeclareMathOperator*{\rowspace}{RS}
\DeclareMathOperator*{\colspace}{CS}
\DeclareMathOperator*{\dom}{dom}
\DeclareMathOperator*{\closure}{cl}
\DeclareMathOperator*{\vol}{vol}
\DeclareMathOperator*{\Span}{span}
\DeclareMathOperator*{\polylog}{polylog}
\DeclareMathOperator*{\Band}{Band}
\newcommand{\grad}{\mathrm{grad}}
\newcommand{\bias}{\mathrm{Bias}}
\newcommand{\var}{\mathrm{Var}}

\newcommand{\Acal}{\ensuremath{\mathcal{A}}}
\newcommand{\Cset}{\ensuremath{\mathcal{C}}}
\newcommand{\X}{\ensuremath{\mathcal{X}}}
\newcommand{\Y}{\ensuremath{\mathcal{Y}}}
\newcommand{\Z}{\ensuremath{\mathcal{Z}}}
\newcommand{\R}{\ensuremath{\mathbb{R}}}
\newcommand{\C}{\ensuremath{\mathbb{C}}}
\newcommand{\Bcal}{\ensuremath{\mathcal{B}}}
\newcommand{\G}{\ensuremath{\mathcal{G}}}

\newcommand{\N}{\ensuremath{\mathbb{N}}}
\newcommand{\F}{\ensuremath{\mathcal{F}}}
\newcommand{\I}{\ensuremath{\mathcal{I}}}
\newcommand{\Set}{\ensuremath{\mathcal{S}}}
\newcommand{\Hyp}{\ensuremath{\mathcal{H}}}
\newcommand{\Loss}{\ensuremath{\mathcal{L}}}
\newcommand{\Lagrange}{\ensuremath{\mathcal{L}}}
\newcommand{\norm}[1]{\lVert #1 \rVert}
\newcommand{\bignorm}[1]{\left\lVert #1 \right\rVert}
\newcommand{\twonorm}[1]{\lVert #1 \rVert_{2}}
\newcommand{\bigtwonorm}[1]{\left\lVert #1 \right\rVert_{2}}
\newcommand{\spectralnorm}[1]{\twonorm{#1}}
\newcommand{\bigspectralnorm}[1]{\bigtwonorm{#1}}
\newcommand{\maxnorm}[1]{\lVert #1 \rVert_{\infty}}
\newcommand{\onenorm}[1]{\left\lVert #1 \right\rVert_{1}}
\newcommand{\mb}[1]{\mathbf{#1}}
\newcommand{\ip}[2]{\ensuremath{\langle #1, #2 \rangle}}
\newcommand{\PD}[2]{\ensuremath{\frac{\partial #1}{\partial #2}}}
\newcommand{\Var}{\mathrm{Var}}
\newcommand{\E}{\mathbb{E}}
\newcommand{\abs}[1]{\ensuremath{| #1 |}}
\newcommand{\bigabs}[1]{\ensuremath{\left| #1 \right|}}
\newcommand{\floor}[1]{\lfloor #1 \rfloor}
\newcommand{\ceil}[1]{\lceil #1 \rceil}
\newcommand{\Normal}{\mathcal{N}}
\newcommand{\rdraw}{\xleftarrow{\$}}
\newcommand{\ind}{\mathbf{1}}
\renewcommand{\vec}{\mathrm{vec}}
\newcommand{\Sym}{\mathbf{S}}

\newcommand{\leb}{\mu}
\renewcommand{\Pr}{\mathbb{P}}
\newcommand{\T}{*}
\newcommand{\Ncal}{\mathcal{N}}
\newcommand{\vecx}{{x}}
\newcommand{\vecw}{{w}}
\newcommand{\vecu}{{u}}
\newcommand{\PP}{\mathbb{P}}
\newcommand{\RR}{\mathbb{R}}
\newcommand{\Res}[1]{\mathfrak{R}_{#1}}


\newcommand{\statedim}{n}
\newcommand{\inputdim}{p}
\newcommand{\hinf}{\mathcal{H}_\infty}
\newcommand{\htwo}{\mathcal{H}_2}

\newcommand{\ltwonorm}[1]{\| #1 \|_2}
\newcommand{\hinfnorm}[1]{\| #1 \|_{\hinf}}
\newcommand{\bighinfnorm}[1]{\left \| #1 \right\|_{\hinf}}
\newcommand{\bightwonorm}[1]{\left \| #1 \right\|_{\htwo}}
\newcommand{\iid}{\stackrel{\mathclap{\text{\scriptsize{ \tiny i.i.d.}}}}{\sim}}
\newcommand{\infnorm}[1]{\left\|#1\right\|_{\infty}}
\newcommand{\linfnorm}[1]{\left\|#1\right\|_{\ell_\infty}}
\newcommand{\LTV}{\mathcal{L}_{\mathrm{TV}}}
\newcommand{\LTI}{\mathcal{L}_{\mathrm{TI}}}
\newcommand{\linffnorm}[1]{\left\|#1\right\|_{\ell_\infty\to\ell_\infty}}
\newcommand{\Sp}{\tf S_+}
\newcommand{\Sm}{\tf S_-}
\newcommand{\RHinf}{\mathcal{RH}_\infty}
\newcommand{\RLinf}{\mathcal{RL}_\infty}

\newcommand{\opt}{\mathrm{opt}}

\def \basefigwidth{0.49}


\title{Robust Performance Guarantees for System Level Synthesis}
\author{Nikolai Matni
\vspace{0.0625in}
\\
Department of Electrical and Systems Engineering\\
University of Pennsylvania}

\date{\today}

\begin{document}
\maketitle
\begin{abstract}
We generalize the System Level Synthesis (SLS) framework to systems defined by bounded causal linear operators, and use this parameterization to make connections between robust SLS and classical results from the robust control literature.  In particular, by leveraging results from $\mathcal{L}_1$ robust control, we show that necessary and sufficient conditions for robust performance with respect to causal bounded linear uncertainty in the system dynamics can be translated into convex constraints on the system responses.  We exploit this connection to show that these conditions naturally allow for the incorporation of delay, sparsity, and locality constraints on the system responses and resulting controller implementation, allowing these methods to be applied to large-scale distributed control problems -- to the best of our knowledge, these are the first such robust performance guarantees for distributed control systems.  %We also show that for suitably small model uncertainty, sub-optimality bounds on robust performance, as measured with respect to performance achieved by an optimal controller for the true model, can be obtained as a function of the size of the model uncertainty, thus providing a natural bridge between tools from robust control and finite-data guarantees for system identification.
\end{abstract}

\section{Introduction}
\label{sec:introduction}
%!TEX root = main.tex
Robust control seeks to explicitly account for the unavoidable mismatch between the predictions made by a mathematical model of a system and the behavior of the system itself.  In the context of linear control systems, robust control techniques \cite{khammash1990stability,dahleh1994control,zhou1996robust} have proven invaluable in applications ranging from process control to aerospace engineering.  A challenge in robust control is the tension between how detailed a description of model uncertainty is available, and the conservativeness of corresponding computationally tractable analysis and synthesis tools.  Indeed, many of the celebrated tools from robust control, such as the structured singular value (see \cite{packard1993complex}), structured small gain theorems (see Ch 7.2 of \cite{dahleh1994control}), and integral quadratic constraints (IQCs) (see \cite{megretski1997system}), seek to optimally navigate this tension.  However, as of yet, few of these results have been extended in a provably non-conservative manner to the large-scale distributed setting.

Although there is a rich and increasingly mature body of work tackling the distributed optimal control of linear systems (see \cite{2006_Rotkowitz_QI_TAC, 2012_Mahajan_Info_survey, wang2019system,zheng2019equivalence}, and references therein), some of which have robust control interpretations with respect to unstructured norm bounded uncertainty (e.g., \cite{langbort2004distributed,matni2014distributed,lessard2014state,rosinger2017structured,ahmadi2018distributed}), to the best of our knowledge no necessary and sufficient conditions for the robust performance of large-scale distributed controllers exist.  Another branch of related work are methods that use dissipativity theory combined with distributed optimization techniques to derive sufficient conditions for the stability of known interconnected systems, see for example \cite{arcak2016networks,meissen2015compositional,anderson2011dynamical} and the references therein -- while applicable to large-scale systems, these methods can often lead to conservative bounds.


In this paper, we address this gap by leveraging the System Level Synthesis (SLS)  framework \cite{anderson2019system,wang2019system}.  SLS reformulates robust and optimal control problems as an optimization over the achievable closed loop behavior, or \emph{system responses}, of a linear-time-invariant (LTI) dynamical system, and in particular shows that it is necessary and sufficient to constrain these system responses to lie in an affine subspace defined by the dynamics.  This parameterization has been successfully exploited in the context of the distributed optimal control of finite-dimensional LTI systems to scale controller synthesis and implementation techniques to systems of arbitrary size under practically realistic assumptions on the underlying system \cite{wang2014localized, wang2016localized,wang2018separable}.  In order to accommodate general linear time varying uncertainty, we extend the SLS parameterization of internally stabilizing controllers to a class of systems described by causal bounded linear operators, and show how this parameterization can be used to make connections to classical robust synthesis techniques \cite{khammash1990stability,dahleh1994control}.  In doing so, we derive necessary and sufficient conditions for robust performance in terms of \emph{convex} constraints on the system response variables.  We then exploit this connection to show that these necessary and sufficient conditions are equally applicable when additional delay, sparsity, and locality constraints are imposed on the system responses and controller implementation.  To the best of our knowledge, these are the first such necessary and sufficient conditions that are applicable to large-scale uncertain systems and distributed controllers.  In particular, our contributions are:
\begin{itemize}
\item A generalization of the SLS parameterization of stabilizing state-feedback controllers for finite-dimensional LTI systems \cite{wang2019system} to systems with dynamics described by bounded and causal linear operators, wherein we show that constraining system responses to lie in an affine subspace defined by the system dynamics is necessary and sufficient for them to be achievable by a causal linear controller;
\item A generalization of the robustness result of \cite{matni2017scalable} that shows that the  generalized SLS parameterization is stable with respect to perturbations away from the aforementioned achievability subspace, as well as an explicit characterization of the effects of these perturbations on the actual closed loop behavior achieved by the correspondingly perturbed controller implementation;
\item The formulation and solution of a robust performance problem in terms of system response variables for a linear-time-invariant dynamical system subject to bounded and causal linear uncertainty that naturally allows for delay, sparsity, and locality structure to be imposed on the system response and corresponding controller implementation;
\end{itemize}

The rest of the paper is structured as follows: in Section 2, we introduce notation and basic definitions of stability and well-posedness.  In Section 3, we review the SLS parameterization for finite-dimensional LTI systems, and generalize it to a richer class of systems with dynamics described by bounded linear operators.  In Section 4, we consider a robust version of the generalized SLS parameterization derived in Section 3, and provide necessary and sufficient conditions for robust performance in terms of \emph{convex} constraints on the system responses.  In Section 4.1, we highlight how these convex constraints can naturally be integrated with structural constraints on the system responses, such as delay, sparsity, and locality constraints, while still preserving the necessity and sufficiency of the identified conditions.  We end with conclusions and discussions of future work in Section 5.

\section{Notation and Preliminaries}
\label{sec:notation}
%!TEX root = main.tex
We slightly adapt the notation used in \cite{khammash1990stability}.  We use Latin letters to denote vectors and matrices, e.g., $Ax = b$, and bold-face Latin letters to denote signals and operators, e.g., $\tf x = (x_t)_{t=0}^\infty$, and $\tf y = \tf G \tf u$.  We let $\ell_\infty$ denote the space of all bounded sequences of real numbers, i.e., $\tf x = (x_t)_{t=0}^\infty$ if and only if $\sup_t |x_t|<\infty$, in which case we define $\linfnorm{\tf x} = \sup_t |x_t|$.  Similarly, we let $\ell^q_\infty$ denote the space of all $q$-tuples of elements of $\ell_\infty$: if $\tf x = (\tf x^1, \dots, \tf x^q) \in \ell^q_\infty$, then $\linfnorm{\tf x} = \max_{i=1,\dots, q}\linfnorm{\tf x^i}$.  We also define the extended space $\ell^{q,e}_\infty$ which is equal to the space of all $q$-tuples of sequences of real numbers.  We let $\tf S_+$ denote the right shift operator such tht if $\tf x = (x_t)_{t=0}^\infty$, then $\tf S_+\tf x = (0,x_0, x_1, \dots)$.  Similarly, we let $\tf S_-$ denote the left shift operator such that $\tf S_- \tf x = (x_1,x_2, \dots)$.  Hence $S_-S_+ = I$, but in general $S_+ S_- \neq I$.  We let $\LTV^{p,q}$ be the space of all bounded linear causal operators mapping $\ell^q_\infty \to \ell^p_\infty$, and broadly refer to all such operators as $\ell_\infty$-stable.  If $\tf R \in \LTV^{p,q}$, then $\linffnorm{\tf R} := \sup_{\linfnorm{x}\leq 1}\linfnorm{\tf R \tf x}$, which is the induced operator norm.  Note that each $\tf R \in \LTV^{p,q}$ can be completely characterized by its block lower-triangular pulse response matrix.  We denote by $\LTI^{p,q}$ the subspace of $\LTV^{p,q}$ consisting of time-invariant operators.

\section{Operator System Level Parameterization}
\label{sec:operator}
%!TEX root = main.tex
Let $\tf A \in \LTV^{n,n}$ and $\tf B \in \LTV^{n,p}$, and consider the dynamical system mapping $\ell^{n,e}_\infty \times \ell^{p,e}_\infty \to \ell^{n,e}_\infty$ defined by
\begin{equation}
\tf x = \Sp\tf A\tf x + \Sp \tf B \tf u + \Sp \tf w.
\label{eq:dynamics}
\end{equation}
As $\Sp\tf A$ is strictly causal, the dynamics \eqref{eq:dynamics} are well posed in the sense that $(I-\Sp\tf A)^{-1}$ exists as an operator from $\ell^{n,e}_\infty \times \ell^{p,e}_\infty \to \ell^{n,e}_\infty$ \cite{dahleh1994control}.  We emphasize that although we impose that $\A$ be $\ell_\infty$-stable, this does not imply that the dynamics \eqref{eq:dynamics} are themselves open-loop stable.  Rather, open-loop stability of the system is determined by the $\ell_\infty$ stability of the operator $(I-\Sp\tf A)^{-1}$.

Note that if $\tf A$ and $\tf B$ are memoryless and time invariant, i.e., if their matrix representations are block-diagonal $\tf A = \mathrm{blkdiag}(A,A,\dots)$, $\tf B = \mathrm{blkdiag}(B,B,\dots)$, then the system \eqref{eq:dynamics} reduce to the familiar finite-dimensional linear time-invariant system
\begin{equation}
x_{t+1} = A_t x_t + Bu_t + w_t, \, x_0 = 0,
\label{eq:lti-dynamics}
\end{equation}
where once again stability of the closed loop system is determined by the stability of the operator $(zI-A)^{-1}$ as opposed to the boundedness of the matrix $A$.

For LTI systems \eqref{eq:lti-dynamics}, the System Level Synthesis (SLS) framework \cite{wang2019system,anderson2019system} provides an appealing parameterization of all closed loop responses from $\tf w \to (\tf x, \tf u)$ achievable by a causal state-feedback LTI control law $\tf K$ such that $\tf u = \tf K \tf x$, as summarized in the following theorem.

\begin{theorem}[Theorem 1, \cite{wang2019system}]\label{thm:lti-sls}
For the LTI system \eqref{eq:lti-dynamics} with causal state-feedback LTI control law $\tf u = \tf K \tf x$, the follwing are true:
\begin{enumerate}
\item The affine subspace defined by 
\begin{equation}
\begin{bmatrix} zI - A & - B \end{bmatrix} \begin{bmatrix} \Phix \\ \Phiu \end{bmatrix} = I \, \ \Phix, \Phiu \in \frac{1}{z}\RHinf
\label{eq:lti-achievable}
\end{equation}
parameterizes all system responses
\begin{equation}
\begin{bmatrix} \tf x \\ \tf u \end{bmatrix} = \begin{bmatrix} \Phix \\ \Phiu \end{bmatrix} \tf w
\label{eq:lti-response}
\end{equation}
achievable by an internally stabilizing state-feedback controller $\tf K$.
\item For any transfer matrices $\left\{\Phix, \Phiu\right\}$ satisfying the constraints \eqref{eq:lti-achievable}, the control signal computed via\footnote{We note that due to the affine constraints \eqref{eq:lti-achievable}, $z\Phix-I$ is strictly causal, and thus feedback loop between $\tf{\hat x}$ and $\tf{\hat w}$ is well posed.}
\begin{equation}
\begin{array}{rcl}
\tf u &=& z\Phiu \tf{\hat w} \\
\tf{\hat w} &=& \tf x - \tf{\hat x} \\ 
\tf{\hat x} &=& (z\Phix - I)\tf{\hat w}
\end{array}
\label{eq:lti-realization}
\end{equation}
is internally stabilizing and achieves the desired response \eqref{eq:lti-response}.
\end{enumerate}
\end{theorem}

Thus, in the case of LTI systems \eqref{eq:lti-dynamics}, the search for an optimal controller $\tf K$ can be converted to a search over system responses $\left\{\Phix,\Phiu\right\}$ constrained to lie in the affine space defined by equation \eqref{eq:lti-achievable}.  This fact, combined with the transparent mapping between the system responses and the corresponding controller implementation \eqref{eq:lti-realization}, has been successfully exploited for the synthesis of distributed optimal controllers for large-scale systems by introducing additional structural constraints on the system responses, and corresponding controller implementation \eqref{eq:lti-realization}, through additional subspace constraints -- we refer the reader to \cite{wang2014localized,wang2016localized,wang2018separable} for more details.

Another favorable feature of the parameterization defined in Theorem \ref{thm:lti-sls} is that it is provably stable under deviations from the subspace \eqref{eq:lti-achievable}, as summarized in the following theorem from \cite{matni2017scalable}.

\begin{theorem}[Theorem 2, \cite{matni2017scalable}]\label{thm:lti-robust}
Let $(\Phixh,\Phiuh,\D)$ be a solution to
\begin{equation}
\begin{bmatrix} zI - A & - B \end{bmatrix} \begin{bmatrix} \Phixh \\ \Phiuh \end{bmatrix} = I - \D \, \ \Phixh, \Phiuh \in \frac{1}{z}\RHinf.
\label{eq:lti-robust}
\end{equation}
If $(I-\D)^{-1}$ exists as on operator in $\RLinf$, then the controller implementation \eqref{eq:lti-realization} defined in terms of the transfer matrices $\left\{\Phixh,\Phiuh\right\}$ achieves the closed loop responses
\begin{equation}
\begin{bmatrix}\tf x \\ \tf u \end{bmatrix} = \begin{bmatrix} \Phixh \\ \Phiuh \end{bmatrix}(I-\D)^{-1}\tf w,
\end{equation}
on the LTI system \eqref{eq:lti-dynamics}, and is internally stabilizing if and only if $(I-\D)^{-1} \in \RHinf$.
\end{theorem}
This parameterization has proved crucial in providing tractable approximations to non-convex distributed control problems \cite{matni2017scalable}, and in providing sub-optimality bounds for robust controllers as applied to learned systems \cite{dean2017sample,dean2019safely}.  However, in these past works, very crude approximations based solely on small gain bounds and triangle inequalities were used to control the effects of the uncertain map $(I-\D)^{-1}$ on system stability and performance.  In this work we show that Theorems \ref{thm:lti-sls} and \ref{thm:lti-robust} can be extended to a more general setting that subsequently allows us to make connections to well developed theory in the robust control literature \cite{khammash1990stability,dahleh1994control}.  Although we focus on $\mathcal{L}_1$ optimal control in this paper due to its favorable separability structure (cf. \S \ref{sec:discussion}), we expect analogous results to carry over naturally in the $\mathcal{H}_\infty$ setting.

\subsection{Necessary Conditions}
Here we characterize a set of affine constraints that the closed loop system responses of system \eqref{eq:dynamics} must satisfy if they are induced by a linear and causal controller $\tf K : \ell_\infty^{n,e} \to \ell_\infty^{p,e}$ via the control law $\tf u = \tf K \tf x$.

\begin{proposition}\label{prop:necessity}
Let $\tf K: \ell_\infty^{n,e} \to \ell_\infty^{p,e}$ be a linear and causal map such that $(I-\Sp(\A+\B\K))^{-1}\Sp\in\LTV^{n,n}$ and $\K(I-\Sp(\A+\B\K))^{-1}\Sp\in\LTV^{n,p}$, i.e., let $\tf K$ be a linear causal and stabilizing controller such that closed loop maps from $\tf w \to (\tf x, \tf u)$ are $\ell_\infty$-stable.  Then all maps taking $\tf w \to (\tf x, \tf u)$ achievable by such a $\tf K$ satisfy the constraints
\begin{equation}
\begin{bmatrix} I-\Sp\A & - \Sp\B\end{bmatrix}\begin{bmatrix} \Phix \\ \Phiu \end{bmatrix} = \Sp, \ \Phix, \Phiu \text{ strictly causal, linear, and $\ell_\infty$-stable.}
\label{eq:osls-achievable}
\end{equation}
\end{proposition}
\begin{proof}
As $\Sp(\A+\B\K)$ is strictly causal, the feedback interconnection is well posed in the sense that $(I-\Sp(\A+\B\K))^{-1}$ exists as a map from $\ell_\infty^{n,e} \to \ell_\infty^{n,e}$.  By assumption, both $(I-\Sp(\A+\B\K))^{-1}$ and $\tf K(I-\Sp(\A+\B\K))^{-1}$ are $\ell_\infty$-stable.  Defining
\[
\begin{bmatrix}
\Phix \\ \Phiu
\end{bmatrix} =
\begin{bmatrix} I \\ \K \end{bmatrix}(I-\Sp(\A+\B\K))^{-1},
\]
it is easily verified that $\left\{\Phix,\Phiu\right\}$ satisfy constraint \eqref{eq:osls-achievable}.
\end{proof}
\begin{remark}
If $\A$ and $\B$ are memoryless and LTI, and $\K$ is LTI, then constraint \eqref{eq:osls-achievable} simplifies to
\begin{equation}
\begin{bmatrix} I - \frac{1}{z}A & -\frac{1}{z}B \end{bmatrix} \begin{bmatrix} \Phix \\ \Phiu \end{bmatrix} = \frac{1}{z}I \, \ \Phix, \Phiu \in \frac{1}{z}\RHinf,
\end{equation}
which is clearly equivalent to \eqref{eq:lti-achievable}.
\end{remark}

\subsection{Controller Implementation}
We now show how to construct an internally stabilizing controller from any operators $\left \{\Phix,\Phiu\right\}$ satisfying constraint \eqref{eq:osls-achievable} that achieves the desired response from $\tf w \to (\tf x, \tf u)$.
\begin{proposition}\label{prop:sufficiency}
Let $\left\{\Phix,\Phiu\right\}$ satisfy constraint \eqref{eq:osls-achievable}.  Then the controller implementation shown in Fig. \ref{fig:realization}, described by the equations
\begin{equation}
\begin{array}{rcl}
\tf u &=& \Phix \Sm \tf{\hat w}\\
\tf{\hat w} & = & x - \tf{\hat x} \\
\tf{\hat x} &=& (\Phix\Sm-I)\tf{\hat w},
\end{array}
\label{eq:realization}
\end{equation}
is well-posed and $\ell_\infty$ stable as a map from $(\tf w, \tf \delta_y, \tf \delta_u) \to (\tf x, \tf u, \tf{\hat w})$ and achieves the desired closed loop response
\begin{equation}
\begin{bmatrix} \tf x \\ \tf u \end{bmatrix} = \begin{bmatrix} \Phix \\ \Phiu \end{bmatrix}\tf w.
\label{eq:response}
\end{equation}
\end{proposition}
\begin{proof}
From equations \eqref{eq:dynamics} and \eqref{eq:realization} (alternatively, from Fig. \ref{fig:realization}), we have that
\begin{equation}
\begin{array}{rcl}
\x &=& \Sp\A\x + \Sp\B\uu + \Sp \w\\
\uu &=& \Phiu\Sm\what + \tf \delta_u\\
\what &=& \x + \tf \delta_y + (I-\Phix\Sm)\what,
\end{array}
\label{eq:internal-stability}
\end{equation}
with $\hat{w}_0 = 0$, i.e., $\what$ a strictly causal signal.  We first observe that the restriction of $(I-\Phix\Sm)$ to strictly causal signals is itself strictly causal, as the constraint \eqref{eq:osls-achievable} enforces that the block lower triangular matrix representation of $\Phix$ has identify matrices along its first block sub-diagonal, i.e., for $\Phix = (\Phi_x(i,j))_{i,j=0}^\infty$ the block lower triangular matrix representation of $\Phix$, we have that $\Phi_x(i,i-1)=I$ for all $i\geq 1$.  It therefore follows that the feedback loop between $\xhat$ and $\what$ is well posed.  By rote calculation, tt follows from equation \eqref{eq:internal-stability} that the closed loop maps from $(\tf w, \tf \delta_y, \tf \delta_u) \to (\tf x, \tf u, \tf{\hat w})$ are given by
\begin{equation}
\begin{bmatrix}
\x \\ \uu \\ \what
\end{bmatrix} =
\begin{bmatrix} \Phix & \Phix(\Sm - \A) & \Phix\B \\
\Phiu & \Phiu(\Sm-\A) & I + \Phiu\B \\
\Sp & I - \Sp\A & \Sp \B
\end{bmatrix} \begin{bmatrix} \tf w \\ \tf \delta_y \\ \tf \delta_u \end{bmatrix}.
\end{equation}
By assumption, $\Phix$, $\Phiu$, $\A$, and $\B$ are all $\ell_\infty$-stable, and hence the interconnection illustrated in Fig. \ref{fig:realization}, and described by equations \eqref{eq:dynamics} and \eqref{eq:internal-stability} is well-posed and $\ell_\infty$-stable.
\end{proof}

\begin{remark}
If $\A$ and $\B$ are memoryless and LTI, and $\K$ is LTI, then so are the system responses $\left\{\Phix,\Phiu\right\}$, and consequently the right and left shift operators $\Sp$ and $\Sm$ become $\frac{1}{z}$ and $z$, respectively, recovering the controller implementation \eqref{eq:lti-realization}.
\end{remark}

\subsection{Robust Operator System Level Synthesis}
Thus Propositions \ref{prop:necessity} and \ref{prop:sufficiency} show that the parameterization of Theorem \ref{thm:lti-sls} can be extended to a class of dynamics described by bounded and causal linear operators in feedback with a causal linear controller.  We now show that this extension enjoys similar stability properties with respect to perturbations from the subspace \eqref{eq:osls-achievable}.

\begin{theorem}\label{thm:robust-sls}
Let $\A\in\LTV^{n,n}$ and $\B\in\LTV^{n,p}$, and suppose that $\left\{\Phixh,\Phiuh\right\}$ satisfy
\begin{equation}
\begin{bmatrix} I-\Sp\A & - \Sp\B\end{bmatrix}\begin{bmatrix} \Phixh \\ \Phiuh \end{bmatrix} = \Sp(I-\D), \ \Phix, \Phiu \text{ strictly causal, linear, and $\ell_\infty$-stable,}
\end{equation}
for $\D$ a strictly causal linear operator from $\ell_\infty^{n,e}\to \ell_\infty^{n,e}$.  Then the controller implementation \eqref{eq:realization} defined in terms of the operators $\left\{\Phixh,\Phiuh\right\}$ is well posed and achieves the following response
\begin{equation}
\begin{bmatrix} \x \\ \uu \end{bmatrix} = \begin{bmatrix}\Phixh\\ \Phiuh \end{bmatrix}(I-\D)^{-1}\w.
\label{eq:robust-response}
\end{equation}
Further, this interconnection is $\ell_\infty$-stable if and only if $(I-\D)^{-1}$ is $\ell_\infty$-stable.
\end{theorem}
\begin{proof}
As $\D$ is strictly causal by assumption, $(I-\D)^{-1}$ eists as a map from $\ell_\infty^{n,e} \to \ell_\infty^{n,e}$.  Going through a similar argument as that in the proof of Proposition \ref{prop:sufficiency}, we observe that
\begin{equation}
\begin{bmatrix}
\x \\ \uu \\ \what
\end{bmatrix} =
\begin{bmatrix} \Phix I_\D & \Phix I_\D(\Sm - \A) & \Phix I_\D \B \\
\Phiu I_\D& \Phiu I_\D(\Sm-\A) & I + \Phiu I_\D\B \\
\Sp I_\D & I_\D(I - \Sp\A) & I_\D\Sp \B
\end{bmatrix} \begin{bmatrix} \tf w \\ \tf \delta_y \\ \tf \delta_u \end{bmatrix},
\end{equation}
where we let $I_\D := (I-\D)^{-1}$.  Thus we see that the desired map \eqref{eq:robust-response} from $\w \to (\x,\uu)$ is achived.  Further, as $\Phix$, $\Phiu$, $\A$, $\B$ are all $\ell_\infty$-stable by assumption, it follows that the $\ell_\infty$-stability of the map from $(\tf w, \tf \delta_y, \tf \delta_u) \to (\x, \uu, \what)$ is determined by the $\ell_\infty$-stability of $I_\D$, from which the result follows.
\end{proof}


\section{Robust Performance under Model Uncertainty}
\label{sec:robust-perf}
%!TEX root = main-ugly.tex
We now use the tools developed in the previous section to identify necessary and sufficient conditions for the robust stability and robust performance of a system \eqref{eq:dynamics} subject to bounded perturbations in its $\A$ and $\B$ operators.  In particular consider the system
\begin{equation}
\tf x = \Sp(\A_0 + \DA)\x + \Sp(\B_0 + \DB)\uu + \Sp \w,
\label{eq:uncertain-dynamics}
\end{equation}
where $\A_0 = \mathrm{blkdiag}(\hat A,\hat A,\dots)$ and $\B_0 = \mathrm{blkdiag}(\hat B, \hat B,\dots)$ are memoryless LTI operators defining a nominal LTI system $x_{t+1} = \hat Ax_t + \hat Bu_t + w_t$, and $\DA$ and $\DB$ are $\ell_\infty$-stable and satisfy 
\begin{equation}
\linffnorm{[\DA,\, \DB]}\leq \eps.
\label{eq:eps-bound}
\end{equation}
We first identify SLS based necessary and sufficient conditions for robust stability, and then build upon those to formulate a robust performance problem.  

We consider the following robust control problem: find a LTI controller $\tf K : \ell^n_{\infty,e} \to \ell^p_{\infty,e}$, using only the nominal dynamics $(\hat A,\hat B)$ and uncertainty bound $\eps$, such that the dynamics \eqref{eq:uncertain-dynamics} in closed loop with the control law $\uu = \K \x$ is $\ell_\infty$-stable for all admissible uncertainty realizations $(\DA,\DB)$ satisfying \eqref{eq:eps-bound}.  To do so, we recognize that for any LTI $\{\Phixh,\Phiuh\}$ satisfying the LTI achievability constraints \eqref{eq:lti-achievable} defined by $(\hat A, \hat B)$, we then have that
\begin{multline}
\begin{bmatrix} I - \Sp\A_0 - \Sp\DA & - \Sp\B_0 - \Sp\DB \end{bmatrix}\begin{bmatrix} \Phixh \\ \Phiuh \end{bmatrix} \\= \begin{bmatrix} I - \Sp\A_0 & - \Sp\B_0 \end{bmatrix}\begin{bmatrix} \Phixh \\ \Phiuh \end{bmatrix}\\ - \begin{bmatrix} \Sp\DA & \Sp\DB \end{bmatrix}\begin{bmatrix} \Phixh \\ \Phiuh \end{bmatrix}\\ = \Sp\left(I - \Sp\begin{bmatrix} \DA & \DB \end{bmatrix}\begin{bmatrix} \Phixh \\ \Phiuh \end{bmatrix} \right)
\end{multline}
where the final inequality follows from the assumption that $\{\Phixh,\Phiuh\}$ satisfy the LTI achievability constraints \eqref{eq:lti-achievable}, or equivalently \eqref{eq:osls-achievable}, defined in terms of the dynamics $(\hat A, \hat B)$.  Noting that
\begin{equation}
\Dh := \Sp\begin{bmatrix} \DA & \DB \end{bmatrix}\begin{bmatrix} \Phixh \\ \Phiuh \end{bmatrix},
\end{equation}
is a strictly causal $\ell_\infty$-stable operator, we conclude by Theorem \ref{thm:robust-sls} that the controller implementation \eqref{eq:realization} defined in terms of the LTI operators $\{\Phixh,\Phiuh\}$ achieves the following closed loop behavior when applied to the uncertain dynamics \eqref{eq:uncertain-dynamics}:
\begin{equation}
\begin{bmatrix}\x \\ \uu \end{bmatrix} = \begin{bmatrix} \Phixh \\ \Phiuh \end{bmatrix}(I-\Dh)^{-1}\tf w.
\label{eq:dhat-response}
\end{equation}
Further this control law is internally stabilizing if and only if $(I-\Dh)^{-1}$ is $\ell_\infty$-stable.  

Define the controlled output signal to be 
\begin{equation}
\tf z = \tf C\x + \tf D\uu,
\label{eq:z-output}
\end{equation}
for $\tf C = \mathrm{blkdiag}(C,C,\dots)$ \& $\tf D = \mathrm{blkdiag}(D,D,\dots)$ user specified cost matrices,\footnote{For simplicity, we assume $\tf C$ and $\tf D$ to be memoryless and LTI, however our results are equally applicable when the controlled output is defined in terms of LTI  filters $\tf C(z)$ and $\tf D(z)$.} and consider the goal of minimizing the $\ell_\infty\to\ell_\infty$ induced gain from $\tf w \to \tf z$ of the uncertain system \eqref{eq:uncertain-dynamics}.  To lighten notation going forward, we let
\begin{multline}
 \Phih := \begin{bmatrix}\Phixh \\ \Phiuh \end{bmatrix}, \, \D := \frac{1}{\eps}\Sp\begin{bmatrix} \DA & \DB \end{bmatrix},\\ \tf Q := \begin{bmatrix} \tf C & \tf D \end{bmatrix}, \, \ Z_{AB} := \begin{bmatrix} zI-\hat A & - \hat B\end{bmatrix}.
\label{eq:notation}
\end{multline}

We can then pose the robust performance problem for a specified performance level $\gamma \geq 0$ as finding LTI operators $\{\Phixh,\Phiuh\}$ that satisfy 
\begin{equation}\label{eq:robust-perf1}
\begin{aligned}
&{ \bignorm{\Q\Phih \left(I-\eps\Sp\D\Phih\right)^{-1} }_{\ell_\infty \to \ell_\infty}} \leq \gamma \\
&Z_{AB}\Phih = I,\ \Phih \in \frac{1}{z}\RHinf, \\
& \left(I-\eps\Sp\D\Phih\right)^{-1} \text{ is $\ell_\infty$-stable}
\end{aligned}
\end{equation}
for all $\D$ satisfying $\linffnorm{\D}\leq 1$, where we have combined equations  \eqref{eq:dhat-response}, \eqref{eq:z-output}, and \eqref{eq:notation} to derive the robust performance bound condition.  It then follows that the robust performance problem \eqref{eq:robust-perf1} is equivalent to finding an LTI operator $\Phih$ satisfying
\begin{equation} \label{eq:robust-perf}
\begin{aligned}
& \linffnorm{\frac{1}{\gamma}\Q\Phih + \frac{1}{\gamma}\Q\Phih\D(I-(\eps\Phih)\D)^{-1}(\eps\Phih)} \leq 1 \\
& Z_{AB}\Phih = I,\ \Phih \in \frac{1}{z}\RHinf,\   (I-\Phih\D)^{-1} \text{ is $\ell_\infty$-stable}\\
\end{aligned}
\end{equation}
for all $\D$ satisfying $\linffnorm{\D}\leq 1$, where we have used that $(I-\Delta\Phih)^{-1} = I + \Delta(I-\Phih\Delta)^{-1}\Phih$ and that $(I-\tf{GH})^{-1}$ is $\ell_\infty$-stable if and only if $(I-\tf{HG})^{-1}$ is $\ell_\infty$-stable (see Proposition 1, \cite{khammash1990stability}) to recast the expression \eqref{eq:robust-perf1} in a form that matches the linear-fractional-transform (LFT) structure studied in \cite{khammash1990stability,dahleh1994control}.

We can therefore leverage the equivalence between robust stability and performance (see Theorem 5.1, \cite{khammash1990stability}) to conclude that $\Phih$ satisfies the robust performance conditions \eqref{eq:robust-perf} for all $\linffnorm{\D}\leq 1$ if and only if the augmented LTI system
\begin{equation}
\tf M = \begin{bmatrix} \frac{1}{\gamma}\Q\Phih & \frac{1}{\gamma}\Q\Phih \\ \eps\Phih & \eps\Phih \end{bmatrix}
\label{eq:bigM}
\end{equation}
is robustly stable for all structured perturbations $\tilde{\D}$ 
\begin{multline}
\tilde{\D} = \mathrm{blkdiag}\left(\D_1,\D_2\right), \, \D_1, \D_2 \in \LTV, \\ \linffnorm{\D_1} \leq 1, \linffnorm{\D_2} \leq 1.
\end{multline}

The necessary and sufficient conditions for robust stability of the resulting two-block problem can be derived as a special case of Theorem 6.3 of \cite{khammash1990stability}.  The particular case of an  augmented LTI system $\tf M$ satisfying $\tf M_{11} = \tf M_{12}$ and $\tf M_{21} = \tf M_{22}$, as is the case for our problem \eqref{eq:bigM}, is addressed in Ch 8.3 of \cite{khammash1990stability}, where a similarly structured augmented system \eqref{eq:bigM} arises in the context of bounding output sensitivity in the presence of output perturbations.  The necessary and sufficient conditions specified in Theorem 6.3 of \cite{khammash1990stability} reduce to the following \emph{convex} constraints on the system response $\Phih$
\begin{equation} \label{eq:constraints}
\begin{aligned}
&Z_{AB}\Phih = I, \, \Phih \in \frac{1}{z}\RHinf, \\
& \linffnorm{\Q\Phih} + \gamma\linffnorm{\eps\Phih} < \gamma.
\end{aligned}
\end{equation}

\begin{remark}
The above results apply equally to the $\ell_1 \to \ell_1$ induced norm, as it is easily checked to equal the $\ell_\infty \to \ell_\infty$ induced norm of the transpose system (see \cite{matni2017scalable}).
\end{remark}
\begin{remark} Although the constraints \eqref{eq:constraints} are in general infinite-dimensional due to the transfer matrix $\Phih$, principled finite-dimensional approximations, some of which enjoy provable sub-optimality guarantees, are available \cite{matni2017scalable,anderson2019system,dean2017sample}.  Further, for the $\mathcal{L}_1$ problem considered here, the resulting optimization problem is quasi-convex, and can be posed as a linear program (LP) for a fixed $\gamma$, thus enjoying favorable computational complexity properties.
\end{remark}
%It then follows that by bisecting on $\gamma$, e.g., by using golden search, we can find a performance level $\gamma$, and corresponding system responses and controller, satisfying $\gamma \leq \gamma_\star + \epsilon$ in $O\log_2(1/\epsilon)$ iterations, for $\gamma_\star$ the smallest $\gamma$ such that the set defined by \eqref{eq:constraints} is non-empty.
%Thus, an LTI controller $\K=\Phiuh\Phixh^{-1}$ is robustly stabilizing for system \eqref{eq:uncertain-dynamics} if and only if the operators $\left\{\Phixh,\Phiuh\right\}$ satisfy



%!TEX root = main-ugly.tex
%In this section, we discuss to extensions to the robust performance results presented in the previous section.

\subsection{Robust Performance Guarantees for Large-Scale Distributed Control}
\label{sec:extensions}
In previous work \cite{wang2014localized,wang2016localized,wang2018separable}, it was shown that for LTI dynamical systems \eqref{eq:lti-dynamics} defined by structured (i.e., sparse) matrices $(A,B)$, imposing \emph{locality constraints} on the system responses $\{\Phix,\Phiu\}$, i.e., imposing that $\{\Phix,\Phiu\} \in \mathcal{S}$, for $\mathcal{S}$ a suitably defined structure inducing subspace constraint, leads to distributed controllers that enjoyed scalable synthesis and implementation complexity.  Although formally defining these concepts is beyond the scope of this paper, we note that such conditions can be easily enforced on the solution of the robust performance conditions \eqref{eq:constraints} by additionally imposing that $\{\Phixh,\Phiuh\}\in\mathcal{S}$.   Under suitable assumptions on the structure of the cost matrices $(C,D)$ (e.g., that $(C,D)$ are block-diagonal), the resulting problem satisfies a notion of \emph{partial separability}, c.f. \S IV of \cite{wang2018separable}, which allows for the problem to be solved at scale using tools from distributed optimization.  

We emphasize that the conditions identified in \eqref{eq:constraints} remain necessary and sufficient when additional structure is imposed on the system responses $\{\Phixh,\Phiuh\}$ so long as the dynamic perturbations $(\DA,\DB)$ remain unstructured.  In particular, Theorem 6.3 of \cite{khammash1990stability} is applicable to any augmented plant $\tf M \in \RHinf$ -- this condition holds true for the augmented plant defined in equation \eqref{eq:bigM} even when any additional constraints are imposed on $\Phih$.  To the best of our knowledge, these are the first such necessary and sufficient conditions for robust performance that are applicable to large-scale distributed systems for which a structured controller can be computed and implemented at scale.  An exciting direction for future work will be to explore the consequences of locality in the system responses $\{\Phixh,\Phiuh\}$ on necessary and sufficient conditions for robust performance when the perturbations $(\DA,\DB)$ are further constrained to respect the topology of the nominal system, as defined by the support of $(\hat A, \hat B)$.
%
%\subsection{Sub-Optimality Guarantees for Robust Control}
%In \cite{dean2017sample}, it was shown\footnote{The result in \cite{dean2018sample} is for a robust $\mathcal{H}_2$ problem, but the derivation goes through nearly as is for $\mathcal{L}_1$ problem considered here.} that an upper bound to the robust performance problem considered here can be computed by solving
%\begin{equation}
%\begin{array}{rl}
%\min_{\tau \in [0,1)}&\frac{1}{1-\tau}\min_{\Phix,\Phiu}\linffnorm{\begin{bmatrix} C & D \end{bmatrix} \begin{bmatrix} \Phixh \\ \Phiuh \end{bmatrix}} \\
%\text{subject to} & \begin{bmatrix} zI-\hat A & - \hat B \end{bmatrix}\begin{bmatrix}\Phixh \\ \Phiuh \end{bmatrix} = I, \ \linffnorm{\begin{bmatrix} \Phixh \\ \Phiuh \end{bmatrix}} \leq \frac{\tau}{\epsilon}, \\
%&\Phixh, \Phiuh \in \frac{1}{z}\RHinf,
%\end{array}
%\end{equation}
%and further, it was shown that $\epsilon$ is such that $\epsilon(1+\linffnorm{\K_\star})\linffnorm{(zI-(A+B\K_\star))^{-1}}\leq 1/5$, for $\K_\star$ the optimal co

\section{Conclusion and Future Work}
\label{sec:conclusion}
%!TEX root = main.tex
In this paper, we generalized the SLS parameterization of LTI $\ell_\infty$-stabilizing controllers, as well as its robust counterpart, to systems described by bounded and causal linear operators.  This extension, along with a simple algebraic transformation, allowed us to leverage tools from $\mathcal{L}_1$ robust control to derive necessary and sufficient conditions for the robust performance of an uncertain system in terms of convex constraints on the system response.  We argued that these conditions remain necessary and sufficient when additional structure, such as that induced by delay, sparsity, and locality constraints, are imposed on the system response, so long as the uncertain elements $(\DA,\DB)$ remained unstructured.  Further, in the case of $\mathcal{L}_1$ optimal control, the resulting robust performance criteria satisfy the partial-separability properties (assuming suitable structural constraints on the cost matrices) needed to apply the distributed synthesis techniques described in \cite{wang2018separable}, thus making our results applicable to large-scale distributed systems. 

More importantly, we believe that the results in this paper open up a wide and exciting range of future research directions, including but not limited to, deriving analogous results for $\mathcal{H}_\infty$ optimal control, revisiting the structured singular value, $\mu$-synthesis, and structured small gain theorems from a system level perspective, and perhaps most exciting, exploring the interplay between closed loop locality constraints and additional structure in the dynamic uncertainty $(\DA,\DB)$.  Further, it is of interest to see if these tighter conditions can be used to derive interpretable bounds on the degradation in performance of a robust controller as a function of the norm bound $\eps$ on the uncertainty $\linffnorm{[\DA,\DB]}$, as coarser and more conservative versions of such bounds have proved crucial in combining machine learning and robust control techniques \cite{dean2017sample,dean2018regret,dean2019safely}.

\begin{small}
\bibliographystyle{abbrvnat}  
\bibliography{references,Distributed} 
\end{small} 
\end{document}