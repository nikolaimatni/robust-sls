%!TEX root = main.tex
In this paper, we generalized the SLS parameterization of LTI $\ell_\infty$-stabilizing controllers, as well as its robust counterpart, to systems described by bounded and causal linear operators.  This extension, along with a simple algebraic transformation, allowed us to leverage tools from $\mathcal{L}_1$ robust control to derive necessary and sufficient conditions for the robust performance of an uncertain system in terms of convex constraints on the system response.  We argued that these conditions remain necessary and sufficient when additional structure, such as that induced by delay, sparsity, and locality constraints, are imposed on the system response, so long as the uncertain elements $(\DA,\DB)$ remained unstructured.  Further, in the case of $\mathcal{L}_1$ optimal control, the resulting robust performance criteria satisfy the partial-separability properties (assuming suitable structural constraints on the cost matrices) needed to apply the distributed synthesis techniques described in \cite{wang2018separable}, thus making our results applicable to large-scale distributed systems. 

More importantly, we believe that the results in this paper open up a wide and exciting range of future research directions, including but not limited to, deriving analogous results for $\mathcal{H}_\infty$ optimal control, revisiting the structured singular value, $\mu$-synthesis, and structured small gain theorems from a system level perspective, and perhaps most exciting, exploring the interplay between closed loop locality constraints and additional structure in the dynamic uncertainty $(\DA,\DB)$.  Further, it is of interest to see if these tighter conditions can be used to derive interpretable bounds on the degradation in performance of a robust controller as a function of the norm bound $\eps$ on the uncertainty $\linffnorm{[\DA,\DB]}$, as coarser and more conservative versions of such bounds have proved crucial in combining machine learning and robust control techniques \cite{dean2017sample,dean2018regret,dean2019safely}.